\chapter{BÀI 4: BẬC PHẢN ỨNG}
    \section{Kết quả thí nghiệm}
    \subsection{Thí nghiệm 1: Tìm bậc theo $Na_{2}S_{2}O{3}$}
        \begin{center}
% Please add the following required packages to your document preamble:


    \begin{tabular}{|c|c|c|c|c|c|} 
    \hline
    \multirow{2}{*}{\textbf{TN}} & \multicolumn{2}{c|}{\textbf{Nồng độ ban đầu (M)}} & \multirow{2}{*}{$\Delta t_{1}$} & \multirow{2}{*}{$\Delta t_{2}$} & \multirow{2}{*}{$\Delta t_{TB}$}  \\ 
    \cline{2-3}
                                 & $Na_{2}S_{2}O_{3}$ & $H_{2}SO_{4}$                &                    &                    &                     \\ 
    \hline
                 1               &       4            &              8               &          108       &         78         &          93         \\ 
    \hline
                 2               &       8            &              8               &           53       &         37         &          45         \\ 
    \hline
                 3               &       28           &              8               &           28       &         28         &          28         \\
    \hline
    \end{tabular}


        \end{center}

- Từ $\Delta t_{TB}$ của TN1 và TN2 xác định m (tính mẫu):\\

- Từ $\Delta t_{TB}$ của TN2 và TN3 xác định $m_{2}$:\\

- Bậc phản ứng theo $Na_{2}S_{2}O_{3} = \frac{m_{1} + m_{2}}{2} = $\\

\subsection{Thí nghiệm 2: Bậc phản ứng theo $H_{2}SO_{4}$}
\begin{center}
    \begin{tabular}{|c|c|c|c|c|c|}
         \hline \textbf{TN}&$[Na_{2}S_{2}O_{3}]$&$[H_{2}SO_{4}]$&$\Delta t_{1}$&$\Delta t_{2}$&$\Delta t_{TB}$\\
         \hline \textbf{1}&4&8&47&49&2\\
         \hline \textbf{2}&8&8&43&30&7\\
         \hline \textbf{3}&16&8&26&42&16\\
         \hline
    \end{tabular}
\end{center}

- Từ $\Delta t_{TB}$ của TN1 và TN2 xác định $m_{1}$\\
- Từ $\Delta t_{TB}$ của TN2 và TN3 xác đinh $m_{2}$\\\
- Bậc phản ứng theo $H_{2}SO_{4} = \frac{n_{1}+n_{2}}{2} = $\\

\newpage 

\section{Trả lời câu hỏi}

\textbf{1.} Trong TN trên, nồng độ của $Na_{2}S_{2}O_{3}$ và của $H_{2}SO_{4}$ đã ảnh hưởng thế nào lên vận tốc phản ứng? Viết lại biểu thức tính vận tốc phản ứng. Xác định bậc của phản ứng.\\

\textbf{2.} Cơ chế phản ứng trên có thể được viết như sau:
\begin{center}
    $H_{2}SO_{4} + Na_{2}S_{2}O_{3} \to Na_{2}SO_{4} + H_{2}S_{2}O_{3}$ (1)\\
    $H_{2}S_{2}O_{3} \to H_{2}SO_{3} + S \downarrow$ \hspace{2.9cm}(2)\\
\end{center}
Dựa vào kết quả TN có thể kết luận phản ứng (1) hay (2) là phản ứng quyết định vận tốc phản ứng tức là phản ứng xảy ra chậm nhất không? Tại sao? Lưu ý trong các TN trên, lượng axit $H_{2}SO_{4}$ luôn luôn dư so với $Na_{2}S_{2}O_{3}$.\\

\textbf{3.} Dựa trên cơ sở của phương pháp TN thì vận tốc xác định được trong các TN trên được xem là vận tốc trung bình hay vận tốc tức thời?\\

\textbf{4.} Thay đổi thứ tự cho $H_{2}SO_{4}$ và $Na_{2}S_{2}O_{3}$ thì bậc phản ứng có thay đổi không? Tại sao?\\